\documentclass[letterpaper]{article}%
\usepackage{lastpage}%
\usepackage{parskip}%
\usepackage{geometry}%
\geometry{left=22.5mm,right=16.1mm,top=48mm,bottom=25mm,headheight=12.5mm,footskip=12.5mm}%
\usepackage{fontspec}%
\usepackage[spanish,activeacute]{babel}%
\usepackage{anyfontsize}%
\usepackage{fancyhdr}%
\usepackage{csquotes}%
\usepackage[ampersand]{easylist}%
\usepackage[style=ieee,backend=biber]{biblatex}%
\usepackage[skins,breakable]{tcolorbox}%
\usepackage{booktabs}%
\usepackage{color}%
\usepackage{tikz}%
\usepackage{tabularx}%
\usepackage{ragged2e}%
\usepackage{graphicx}%
\usepackage{xcolor}%
%
\setmainfont{Arial}%
\addbibresource{../bibliografia.bib}%
\renewcommand*{\bibfont}{\fontsize{10}{14}\selectfont}%

\defbibenvironment{bibliography}
    {\list
    {\printfield[labelnumberwidth]{labelnumber}}
    {\setlength{\leftmargin}{4cm}
    \setlength{\rightmargin}{1.1cm}
    \setlength{\itemindent}{0pt}
    \setlength{\itemsep}{\bibitemsep}
    \setlength{\parsep}{\bibparsep}}}
    {\endlist}
{\item}
%

\newenvironment{textoMargen}
    {%
    \begin{list}{}{%
        \setlength{\leftmargin}{3.6cm}%
        \setlength{\rightmargin}{1.1cm}%
    }%
    \item[]%
  }
  {%
    \end{list}%
  }
%
\definecolor{gris}{rgb}{0.27,0.27,0.27}%
\definecolor{parte}{rgb}{0.02,0.204,0.404}%
\definecolor{azulsuaveTEC}{rgb}{0.02,0.455,0.773}%
\definecolor{fila}{rgb}{0.929,0.929,0.929}%
\definecolor{linea}{rgb}{0.749,0.749,0.749}%
\fancypagestyle{headfoot}{%
\renewcommand{\headrulewidth}{0pt}%
\renewcommand{\footrulewidth}{0pt}%
\fancyhead{%
}%
\fancyfoot{%
}%
\fancyhead[L]{%
\begin{minipage}{0.5\textwidth}%
\flushleft%
\includegraphics[width=62.5mm]{../figuras/Logo.png}%
\end{minipage}%
}%
\fancyfoot[R]{%
\textcolor{black}{%
Página \thepage \hspace{1pt} de \pageref*{LastPage}%
}%
}%
}%
%
\begin{document}%
\normalsize%
\thispagestyle{empty}%
\begin{tikzpicture}[overlay,remember picture]%
\node[inner sep = 0mm,outer sep = 0mm,anchor = north west,xshift = -23mm,yshift = 22mm] at (0.0,0.0) {\includegraphics[width=21cm]{../figuras/Logo_portada.png}};%
\end{tikzpicture}%
\vspace*{100mm}%
\par\fontsize{14}{0}\selectfont \textcolor{black}{Programa del curso EE{-}4801}%
\par\fontsize{18}{25}\selectfont \textbf{\textcolor{black}{Sistemas eléctricos de transmisión y distribución}}%
\vspace*{15mm}%
\newline%
\begin{tabularx}{\textwidth}{m{0.02\textwidth}m{0.98\textwidth}}%
&\hspace*{0mm}\fontsize{12}{0}\selectfont \textbf{\textcolor{gris}{Escuela de Ingeniería Electromecánica}}\\%
[-12pt]%
&\hspace*{0mm}\fontsize{12}{0}\selectfont \textbf{\textcolor{gris}{Carrera de Ingeniería Electromecánica con énfasis en Instalaciones Electromecánicas}}\\%
\end{tabularx}%
\newpage%
\pagestyle{headfoot}%
\par\fontsize{14}{0}\selectfont \textbf{\textcolor{parte}{I parte: Aspectos relativos al plan de estudios}}%
\par\hspace*{2mm}\fontsize{12}{14}\selectfont \textbf{\textcolor{parte}{1. Datos generales}}%
\vspace*{3mm}%
\newline%
\fontsize{10}{12}\selectfont %
\begin{tabularx}{\textwidth}{p{6cm}p{10cm}}%
\textbf{Nombre del curso:}&Sistemas eléctricos de transmisión y distribución\\%
[10pt]%
\textbf{Código:}&EE{-}4801\\%
[10pt]%
\textbf{Tipo de curso:}&Teórico\\%
[10pt]%
\textbf{Obligatorio o electivo:}&Obligatorio\\%
[10pt]%
\textbf{Nº de créditos:}&3\\%
[10pt]%
\textbf{Nº horas de clase por semana:}&4\\%
[10pt]%
\textbf{Nº horas extraclase por semana:}&5\\%
[10pt]%
\textbf{Ubicación en el plan de estudios:}&Curso de 8\textsuperscript{vo} semestre en Ingeniería Electromecánica con énfasis en Instalaciones Electromecánicas\\%
[10pt]%
\textbf{Requisitos:}&Ninguno\\%
[10pt]%
\textbf{Correquisitos:}&EE{-}0802 Máquinas eléctricas II\\%
[10pt]%
\textbf{El curso es requisito de:}&EE{-}4901 Sistemas de generación y almacenamiento de energía\\%
[10pt]%
\textbf{Asistencia:}&Libre\\%
[10pt]%
\textbf{Suficiencia:}&Si\\%
[10pt]%
\textbf{Posibilidad de reconocimiento:}&Si\\%
[10pt]%
\textbf{Aprobación y actualización del programa:}&01/01/2026 en sesión de Consejo de Escuela 01{-}2026\\%
[10pt]%
\end{tabularx}%
\newpage%
\begin{tabularx}{\textwidth}{p{3cm}p{13cm}}%
\par\fontsize{12}{14}\selectfont \textbf{\textcolor{parte}{2. Descripción general}}&El curso de \emph{Sistemas eléctricos de transmisión y distribución} colabora en el desarrollo de los siguientes rasgos del plan de estudios: comprender los fundamentos de los sistemas de distribución y transmisión de energía eléctrica. \newline\newline Los aprendizajes que los estudiantes desarrollarán en el curso son: describir la estructura y los desafíos de los sistemas de transmisión y distribución de energía eléctrica, destacando su importancia en la red eléctrica; representar los componentes fundamentales de los sistemas eléctricos de potencia, incluyendo generadores, transformadores, líneas de transmisión y elementos de compensación; determinar los métodos de análisis para evaluar el desempeño de las redes eléctricas, tales como flujos de potencia y estudios de cortocircuito; examinar el impacto de tecnologías emergentes como generación distribuida, almacenamiento de energía y redes inteligentes en la operación de los sistemas eléctricos; y identificar las condiciones anormales en la red eléctrica y los mecanismos de protección para garantizar la seguridad y estabilidad operativa del sistema.. \newline\newline Para desempeñarse adecuadamente en este curso, los estudiantes deben poner en práctica lo aprendido en los cursos de: Máquinas eléctricas II, y Laboratorio de máquinas eléctricas II. \newline\newline Una vez aprobado este curso, los estudiantes podrán emplear algunos de los aprendizajes adquiridos en los cursos de: Sistemas de generación y almacenamiento de energía, y Gestión de la energía. \\%
\end{tabularx}%
\vspace*{4mm}%
\newline%
\begin{tabularx}{\textwidth}{p{3cm}p{13cm}}%
\par\fontsize{12}{14}\selectfont \textbf{\textcolor{parte}{3. Objetivos}}&Al final del curso la persona estudiante será capaz de:\newline\newline \textbf{Objetivo general}\begin{itemize}\item Analizar el funcionamiento, modelado y operación de los sistemas de transmisión y distribución de energía eléctrica, considerando sus elementos fundamentales, métodos de análisis y tecnologías emergentes para su gestión eficiente y segura\end{itemize} \vspace{2mm}\textbf{Objetivos específicos}\begin{itemize}\item Describir la estructura y los desafíos de los sistemas de transmisión y distribución de energía eléctrica, destacando su importancia en la red eléctrica\item Representar los componentes fundamentales de los sistemas eléctricos de potencia, incluyendo generadores, transformadores, líneas de transmisión y elementos de compensación\item Determinar los métodos de análisis para evaluar el desempeño de las redes eléctricas, tales como flujos de potencia y estudios de cortocircuito\item Examinar el impacto de tecnologías emergentes como generación distribuida, almacenamiento de energía y redes inteligentes en la operación de los sistemas eléctricos\item Identificar las condiciones anormales en la red eléctrica y los mecanismos de protección para garantizar la seguridad y estabilidad operativa del sistema.\end{itemize}\\%
\end{tabularx}%
\newpage%
\begin{tabularx}{\textwidth}{p{3cm}p{13cm}}%
\par\fontsize{12}{14}\selectfont \textbf{\textcolor{parte}{4. Contenidos}}&En el curso se desarrollaran los siguientes temas:\\%
\end{tabularx}%
\newline%
\par \setlength{\leftskip}{4cm} \begin{easylist} \ListProperties(Progressive*=3ex)

& Introducción

&& Introducción al análisis de los sistemas eléctricos de potencia

&& Perspectivas generales de los sistemas de transmisión y distribución de energía eléctrica 

&& Desafíos y oportunidades

& Elementos de modelado de sistemas eléctricos de potencia

&& Centrales eléctricas 

&& Generador sincrónico: modelo y parámetros

&& Transformador de potencia: modelo y parámetros

&& Elementos de compensación de potencia reactiva

&& Subestaciones eléctricas

&& Cargas estáticas y dinámicas

&& Sistema p.u.

& Elementos estáticos de las redes eléctricas

&& Líneas aéreas

&& Líneas subterráneas

&& Resistencia, inductancia, capacitancia y conductancia de la línea de transmisión.

&& Modelado de la línea de transmisión 

& Modelado de las líneas de transmisión

&& Modelo de la línea corta

&& Modelo de la línea media

&& Modelo de la línea larga

& Modelado de componentes de los sistemas de distribución

&& Líneas eléctricas aéreas y subterráneas

&& Modelado de transformadores 

&& Modelado de la demanda 

&& Modelo de la generación distribuida e inversores 

&& Modelado de sistemas de almacenamiento 

&& Modelado de vehículos eléctricos  

& Métodos de flujos de potencia

&& Planteamiento del problema de flujo de potencia

&& Métodos numéricos para la solución de flujos de potencia

&& Gauss Seidel

&& Newton Rapson

&& Desacoplado

&& Flujo de potencia DC

&& Reconocimiento de programas de simulación

&& Programación en programas de simulación 

& Análisis de Fallas y estudio de cortocircuito

&& Planteamiento del problema de cortocircuito

&& Componentes simétricos 

&& Síntesis de vectores desequilibrados a partir de componentes simétricos

&& Operadores, componentes simétricos de vectores asimétricos

&& Potencia en función de los componentes simétricos

&& Impedancias de secuencia, redes de secuencia

&& Fallas NO Simétricas 

&& Análisis de fallas monofásicas, bifásicas y trifásicas

&& Interpretación de las redes de secuencia interconectadas

&& Análisis de fallas asimétricas utilizando la matriz de impedancias de barras

Operación de sistemas de redes inteligentes de distribución

&& Sistemas avanzados para la gestión de la distribución 

&& Flujos de potencia y el estimador de estados 

&& Control de tensión y potencia reactiva 

&& Análisis de cortocircuito simétrico y asimétrico

&& Esquemas de protección y gestión de fallas

&& Coordinación de recursos energéticos distribuidos

&& Coordinación operativa entre transmisión y distribución

& Análisis de condiciones anormales en la red

&& Sobretensiones por descargas atmosféricas 

&& Origen de las descargas atmosféricas, campo eléctrico

&& Variaciones del campo eléctrico y formación de rayos

&& Pararrayos y zonas de protección, modelo electro-geométrico

&& Parámetros eléctricos de las descargas, energía descargada

&& Blindaje

&& Sobretensiones por maniobra

&& Sobretensiones temporales, sobretensiones de frente lento

&& Sobretensiones de frente rápido, sobretensiones de frente muy rápido

&& Mecanismos de protección, perturbaciones de baja frecuencia

&& Perturbaciones de alta frecuencia, problemas por baja tensión

& Conceptos y aplicaciones de las redes inteligentes

&& Comunicaciones eléctricas y ciberseguridad  

&& Manejo de demanda dinámica y medidores inteligentes 

&& Automatización de sistemas de distribución 

&& Sistemas de almacenamiento y electrónica de potencia

\end{easylist} \setlength{\leftskip}{0cm} %
\newpage%
\par\fontsize{14}{0}\selectfont \textbf{\textcolor{parte}{II parte: Aspectos operativos}}%
\vspace*{4mm}%
\newline%
\fontsize{10}{12}\selectfont %
\begin{tabularx}{\textwidth}{p{3cm}p{13cm}}%
\par\fontsize{12}{14}\selectfont \textbf{\textcolor{parte}{5. Metodología}}&En este curso, la metodología adoptada es constructivista enfocada en la investigación práctica aplicada. Mediante el uso de técnicas como:  estudio de casos, análisis de escenarios y simulación, se espera que el estudiante domine la aplicabilidad de los sistemas de transmisión y distribución.\newline\newline \textbf{Las personas estudiantes podrán desarrollar actividades en las que:} \newline\begin{itemize}\item Analizarán casos de estudio sobre la estructura y operación de los sistemas de transmisión y distribución de energía eléctrica.\item Modelarán componentes de los sistemas eléctricos de potencia utilizando software de simulación.\item Resolverán problemas numéricos de flujos de potencia y cortocircuito aplicando diferentes métodos de análisis.\item Examinarán fallas en la red mediante la interpretación de redes de secuencia y análisis de condiciones anormales.\item Interpretarán resultados de estudios eléctricos para proponer estrategias de mejora en la operación del sistema.\item Comparararán diferentes modelos de líneas de transmisión y distribución para identificar sus ventajas y limitaciones.\item Investigarán tendencias actuales en tecnologías de transmisión, distribución y automatización de redes eléctricas.\item Presentarán informes técnicos sobre los resultados obtenidos en las simulaciones y análisis realizados.\end{itemize}\vspace*{2mm}Este enfoque metodológico permitirá a la persona estudiante analizar el funcionamiento, modelado y operación de los sistemas de transmisión y distribución de energía eléctrica, considerando sus elementos fundamentales, métodos de análisis y tecnologías emergentes para su gestión eficiente y segura\vspace*{2mm} \newline  Si un estudiante requiere apoyos educativos, podrá solicitarlos a través del Departamento de Orientación y Psicología. \newline \\%
\end{tabularx}%
\vspace*{2mm}%
\newline%
\begin{tabularx}{\textwidth}{p{3cm}p{13cm}}%
\par\fontsize{12}{14}\selectfont \textbf{\textcolor{parte}{6. Evaluación}}&La evaluación se distribuye en los siguientes rubros: \newline \begin{itemize} \item Examanes parciales: evaluaciones formales que miden el nivel de comprensión y aplicación de los conceptos clave del curso. Generalmente cubren una parte significativa del contenido visto hasta la fecha y pueden incluir problemas teóricos y prácticos. \item Pruebas cortas: evaluaciones breves y frecuentes que sirven para comprobar el dominio de temas específicos. Suelen ser de menor peso en la calificación final y permiten reforzar el aprendizaje continuo. \item Proyecto: actividad integradora donde los estudiantes aplican conocimientos teóricos y prácticos para resolver un problema real o simulado. Fomenta el desarrollo de habilidades analíticas, de investigación y trabajo en equipo. \end{itemize}\\%
\end{tabularx}%
\vspace*{2mm}%
\newline%
 \begin{minipage}{\linewidth}  \centering  \begin{tabular}{ p{4cm}  p{1.5cm} }  \toprule  Examanes parciales & 60 \% \\  \midrule  Pruebas cortas & 20 \% \\  \midrule  Proyecto & 20 \% \\  \midrule Total & 100 \% \\  \bottomrule  \end{tabular} \end{minipage}%
\vspace*{2mm}%
\newline%
\begin{tabularx}{\textwidth}{p{3cm}p{13cm}}%
&De conformidad con el artículo 78 del Reglamento del Régimen Enseñanza-Aprendizaje del Instituto Tecnológico de Costa Rica y sus Reformas, en este curso la persona estudiante \textbf{no} tiene derecho a presentar un examen de reposición.\\%
\end{tabularx}%
\vspace*{4mm}%
\newline%
\begin{tabularx}{\textwidth}{p{3cm}p{13cm}}%
\par\fontsize{12}{14}\selectfont \textbf{\textcolor{parte}{7. Bibliografía}}&\nocite{grainger1996} \nocite{kothari2008} \nocite{wildi2007} \nocite{kersting2012} \nocite{short2014} \nocite{gonen2008} \nocite{momoh2012} \nocite{ekanayake2012} \nocite{pinheiro2018} \nocite{migliavacca2020} \\%
\end{tabularx}%
\vspace*{-8mm}\printbibliography[heading=none]%
\begin{tabularx}{\textwidth}{p{3cm}p{13cm}}%
\par\fontsize{12}{14}\selectfont \textbf{\textcolor{parte}{8. Persona docente}}&El curso será impartido por:\\%
\end{tabularx}%
\vspace*{-4mm}\begin{textoMargen}\textbf{Dr.{-}Ing. Gustavo Gomez Ramirez} \newline Maestría académica en Ingeniería Eléctrica. Universidad de Costa Rica. Costa Rica. \newline \newline  Maestría Profesional en Administración de Negocios. Universidad Estatal a Distancia. Costa Rica \newline \newline  Doctor en Ingeniería. Licenciado en Ingeniería en Mantenimiento Industrial. Instituto Tecnológico de Costa Rica. Costa Rica. Instituto Tecnológico de Costa Rica. Costa Rica \newline \newline \emph{Correo:} ggomez@itcr.ac.cr\emph{  Teléfono:} 25509354 \vspace*{1mm} \newline \emph{  Oficina:} 17\emph{  Escuela:} Ingeniería Electromecánica\emph{  Sede:} Cartago \vspace*{4mm} \newline \end{textoMargen}%
\end{document}