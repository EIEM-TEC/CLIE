\documentclass[letterpaper]{article}%
\usepackage{lastpage}%
\usepackage{parskip}%
\usepackage{geometry}%
\geometry{left=22.5mm,right=16.1mm,top=48mm,bottom=25mm,headheight=12.5mm,footskip=12.5mm}%
\usepackage{fontspec}%
\usepackage[spanish,activeacute]{babel}%
\usepackage{anyfontsize}%
\usepackage{fancyhdr}%
\usepackage{csquotes}%
\usepackage[ampersand]{easylist}%
\usepackage[style=ieee,backend=biber]{biblatex}%
\usepackage[skins,breakable]{tcolorbox}%
\usepackage{color}%
\usepackage{tikz}%
\usepackage{tabularx}%
\usepackage{ragged2e}%
\usepackage{graphicx}%
\usepackage{xcolor}%
%
\setmainfont{Arial}%
\addbibresource{../bibliografia.bib}%
\renewcommand*{\bibfont}{\fontsize{12}{16}\selectfont}%
\definecolor{gris}{rgb}{0.27,0.27,0.27}%
\definecolor{parte}{rgb}{0.02,0.204,0.404}%
\definecolor{azulsuaveTEC}{rgb}{0.02,0.455,0.773}%
\definecolor{fila}{rgb}{0.929,0.929,0.929}%
\definecolor{linea}{rgb}{0.749,0.749,0.749}%
\fancypagestyle{headfoot}{%
\renewcommand{\headrulewidth}{0pt}%
\renewcommand{\footrulewidth}{0pt}%
\fancyhead{%
}%
\fancyfoot{%
}%
\fancyhead[L]{%
\begin{minipage}{0.5\textwidth}%
\flushleft%
\includegraphics[width=62.5mm]{../figuras/Logo.png}%
\end{minipage}%
}%
\fancyfoot[L]{%
\textcolor{azulsuaveTEC}{%
Escuela de Ingeniería Electromecánica%
}%
\par \parbox{0.85\textwidth}{%
\fontsize{8}{0}\selectfont \textcolor{azulsuaveTEC}{Carrera de Ingeniería Electromecánica}%
}%
}%
\fancyfoot[R]{%
\textcolor{azulsuaveTEC}{%
Página \thepage \hspace{1pt} de \pageref*{LastPage}%
}%
}%
}%
%
\begin{document}%
\normalsize%
\thispagestyle{empty}%
\begin{tikzpicture}[overlay,remember picture]%
\node[inner sep = 0mm,outer sep = 0mm,anchor = north west,xshift = -23mm,yshift = 22mm] at (0.0,0.0) {\includegraphics[width=21cm]{../figuras/Logo_portada.png}};%
\end{tikzpicture}%
\vspace*{100mm}%
\par\fontsize{14}{0}\selectfont \textcolor{black}{Programa del curso MI0712}%
\par\fontsize{18}{25}\selectfont \textbf{\textcolor{azulsuaveTEC}{Modelado y simulación de sistemas electromecánicos}}%
\vspace*{15mm}%
\newline%
\begin{tabularx}{\textwidth}{m{0.02\textwidth}m{0.98\textwidth}}%
&\hspace*{0mm}\fontsize{12}{0}\selectfont \textbf{\textcolor{gris}{Escuela de Ingeniería Electromecánica}}\\%
[-12pt]%
&\hspace*{0mm}\fontsize{12}{0}\selectfont \textbf{\textcolor{gris}{Carrera de Ingeniería Electromecánica}}\\%
\end{tabularx}%
\newpage%
\pagestyle{headfoot}%
\par\fontsize{14}{0}\selectfont \textbf{\textcolor{parte}{I parte: Aspectos relativos al plan de estudios}}%
\par\hspace*{2mm}\fontsize{12}{14}\selectfont \textbf{\textcolor{parte}{1. Datos generales}}%
\vspace*{3mm}%
\newline%
\fontsize{10}{12}\selectfont %
\begin{tabularx}{\textwidth}{p{6cm}p{10cm}}%
\textbf{Nombre del curso:}&Modelado y simulación de sistemas electromecánicos\\%
[10pt]%
\textbf{Código:}&MI0712\\%
[10pt]%
\textbf{Tipo de curso:}&Teórico {-} Práctico\\%
[10pt]%
\textbf{Obligatorio o electivo:}&Obligatorio\\%
[10pt]%
\textbf{Nº de créditos:}&3\\%
[10pt]%
\textbf{Nº horas de clase por semana:}&4\\%
[10pt]%
\textbf{Nº horas extraclase por semana:}&5\\%
[10pt]%
\textbf{Ubicación en el plan de estudios:}&Curso de V semestre en Ingeniería Electromecánica\\%
[10pt]%
\textbf{Requisitos:}&MI3117 Dinámica; CM3207 Métodos numéricos para ingeniería\\%
[10pt]%
\textbf{Correquisitos:}&Ninguno\\%
[10pt]%
\textbf{El curso es requisito de:}&MI3108 Mecánica de fluidos; MI0720 Control automático\\%
[10pt]%
\textbf{Asistencia:}&Obligatoria\\%
[10pt]%
\textbf{Suficiencia:}&Si\\%
[10pt]%
\textbf{Posibilidad de reconocimiento:}&Si\\%
[10pt]%
\textbf{Aprobación y actualización del programa:}&I semestre de 2026\\%
[10pt]%
\end{tabularx}%
\newpage%
\begin{tabularx}{\textwidth}{p{3cm}p{13cm}}%
\par\fontsize{12}{14}\selectfont \textbf{\textcolor{parte}{2. Descripción general}}&El curso de Modelado y Simulación de Sistemas Electromecánicos contribuye significativamente al desarrollo profesional de los estudiantes, ya que proporciona las herramientas necesarias para analizar, representar y comprender sistemas complejos en el ámbito de la ingeniería electromecánica. Este curso fomenta la integración de conocimientos teóricos con herramientas computacionales avanzadas, promoviendo soluciones innovadoras y eficientes.
\newline%
\newline%
Entre los aprendizajes más destacados se encuentran: aplicar técnicas de modelado matemático para representar sistemas electromecánicos; implementar simulaciones utilizando software especializado; analizar los resultados de las simulaciones para optimizar el desempeño de los sistemas; y utilizar modelos para evaluar diferentes escenarios operativos en sistemas electromecánicos.
\newline%
\newline%
Este curso se complementa con Control Automático y Control por Eventos Discretos, sentando las bases para la formación en automática. Juntos, estos cursos permiten a los estudiantes abordar con éxito el diseño y análisis de sistemas de control integrados, esenciales para aplicaciones avanzadas en la ingeniería electromecánica.\\%
\end{tabularx}%
\vspace*{4mm}%
\newline%
\begin{tabularx}{\textwidth}{p{3cm}p{13cm}}%
\par\fontsize{12}{14}\selectfont \textbf{\textcolor{parte}{3. Objetivos}}&Al final del curso la persona estudiante será capaz de:\newline\newline \textbf{Objetivo general}\begin{itemize}\item Desarrollar en los estudiantes las competencias necesarias para modelar y simular sistemas electromecánicos, utilizando herramientas matemáticas y computacionales que permitan analizar y optimizar su comportamiento en diferentes escenarios operativos.\end{itemize} \vspace{2mm}\textbf{Objetivos específicos}\begin{itemize}\item Aplicar técnicas de modelado matemático para representar sistemas electromecánicos, integrando conceptos físicos y de ingeniería que describan su dinámica y funcionamiento.\item Implementar simulaciones computacionales que permitan analizar el comportamiento de los sistemas electromecánicos bajo distintas condiciones operativas.\item Interpretar y evaluar los resultados de las simulaciones para optimizar diseños y mejorar el desempeño de los sistemas electromecánicos.\end{itemize}\\%
\end{tabularx}%
\newpage%
\begin{tabularx}{\textwidth}{p{3cm}p{13cm}}%
\par\fontsize{12}{14}\selectfont \textbf{\textcolor{parte}{4. Contenidos}}&En el curso se desarrollaran los siguientes temas:\\%
\end{tabularx}%


\setlength{\leftskip}{4cm}\begin{easylist}\ListProperties(Progressive*=3ex)

& Conceptos básicos

&& uno

&& dos

& Modelado de sistemas lineales

\end{easylist}\setlength{\leftskip}{0pt}%
\end{document}