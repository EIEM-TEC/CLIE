\documentclass[letterpaper]{article}%
\usepackage{lastpage}%
\usepackage{parskip}%
\usepackage{geometry}%
\geometry{left=22.5mm,right=16.1mm,top=48mm,bottom=25mm,headheight=12.5mm,footskip=12.5mm}%
\usepackage{fontspec}%
\usepackage[spanish,activeacute]{babel}%
\usepackage{anyfontsize}%
\usepackage{fancyhdr}%
\usepackage{csquotes}%
\usepackage[ampersand]{easylist}%
\usepackage[style=ieee,backend=biber]{biblatex}%
\usepackage[skins,breakable]{tcolorbox}%
\usepackage{booktabs}%
\usepackage{color}%
\usepackage{tikz}%
\usepackage{tabularx}%
\usepackage{ragged2e}%
\usepackage{graphicx}%
\usepackage{xcolor}%
%
\setmainfont{Arial}%
\addbibresource{../bibliografia.bib}%
\renewcommand*{\bibfont}{\fontsize{10}{14}\selectfont}%

\defbibenvironment{bibliography}
    {\list
    {\printfield[labelnumberwidth]{labelnumber}}
    {\setlength{\leftmargin}{4cm}
    \setlength{\rightmargin}{1.1cm}
    \setlength{\itemindent}{0pt}
    \setlength{\itemsep}{\bibitemsep}
    \setlength{\parsep}{\bibparsep}}}
    {\endlist}
{\item}
%

\newenvironment{textoMargen}
    {%
    \begin{list}{}{%
        \setlength{\leftmargin}{3.6cm}%
        \setlength{\rightmargin}{1.1cm}%
    }%
    \item[]%
  }
  {%
    \end{list}%
  }
%
\definecolor{gris}{rgb}{0.27,0.27,0.27}%
\definecolor{parte}{rgb}{0.02,0.204,0.404}%
\definecolor{azulsuaveTEC}{rgb}{0.02,0.455,0.773}%
\definecolor{fila}{rgb}{0.929,0.929,0.929}%
\definecolor{linea}{rgb}{0.749,0.749,0.749}%
\fancypagestyle{headfoot}{%
\renewcommand{\headrulewidth}{0pt}%
\renewcommand{\footrulewidth}{0pt}%
\fancyhead{%
}%
\fancyfoot{%
}%
\fancyhead[L]{%
\begin{minipage}{0.5\textwidth}%
\flushleft%
\includegraphics[width=62.5mm]{../figuras/Logo.png}%
\end{minipage}%
}%
\fancyfoot[R]{%
\textcolor{black}{%
Página \thepage \hspace{1pt} de \pageref*{LastPage}%
}%
}%
}%
%
\begin{document}%
\normalsize%
\thispagestyle{empty}%
\begin{tikzpicture}[overlay,remember picture]%
\node[inner sep = 0mm,outer sep = 0mm,anchor = north west,xshift = -23mm,yshift = 22mm] at (0.0,0.0) {\includegraphics[width=21cm]{../figuras/Logo_portada.png}};%
\end{tikzpicture}%
\vspace*{100mm}%
\par\fontsize{14}{0}\selectfont \textcolor{black}{Programa del curso EE{-}0503}%
\par\fontsize{18}{25}\selectfont \textbf{\textcolor{black}{Sistemas analógicos}}%
\vspace*{15mm}%
\newline%
\begin{tabularx}{\textwidth}{m{0.02\textwidth}m{0.98\textwidth}}%
&\hspace*{0mm}\fontsize{12}{0}\selectfont \textbf{\textcolor{gris}{Escuela de Ingeniería Electromecánica}}\\%
[-12pt]%
&\hspace*{0mm}\fontsize{12}{0}\selectfont \textbf{\textcolor{gris}{Carrera de Ingeniería Electromecánica}}\\%
\end{tabularx}%
\newpage%
\pagestyle{headfoot}%
\par\fontsize{14}{0}\selectfont \textbf{\textcolor{parte}{I parte: Aspectos relativos al plan de estudios}}%
\par\hspace*{2mm}\fontsize{12}{14}\selectfont \textbf{\textcolor{parte}{1. Datos generales}}%
\vspace*{3mm}%
\newline%
\fontsize{10}{12}\selectfont %
\begin{tabularx}{\textwidth}{p{6cm}p{10cm}}%
\textbf{Nombre del curso:}&Sistemas analógicos\\%
[10pt]%
\textbf{Código:}&EE{-}0503\\%
[10pt]%
\textbf{Tipo de curso:}&Teórico {-} Práctico\\%
[10pt]%
\textbf{Obligatorio o electivo:}&Obligatorio\\%
[10pt]%
\textbf{Nº de créditos:}&2\\%
[10pt]%
\textbf{Nº horas de clase por semana:}&4\\%
[10pt]%
\textbf{Nº horas extraclase por semana:}&2\\%
[10pt]%
\textbf{Ubicación en el plan de estudios:}&Curso de 5\textsuperscript{to} semestre en Ingeniería Electromecánica\\%
[10pt]%
\textbf{Requisitos:}&EE{-}0303 Análisis de circuitos I\\%
[10pt]%
\textbf{Correquisitos:}&Ninguno\\%
[10pt]%
\textbf{El curso es requisito de:}&EE{-}0303 Sistemas digitales; EE{-}0303 Control automático\\%
[10pt]%
\textbf{Asistencia:}&Libre\\%
[10pt]%
\textbf{Suficiencia:}&No\\%
[10pt]%
\textbf{Posibilidad de reconocimiento:}&Si\\%
[10pt]%
\textbf{Aprobación y actualización del programa:}&01/01/2026 en sesión de Consejo de Escuela 01{-}2026\\%
[10pt]%
\end{tabularx}%
\newpage%
\begin{tabularx}{\textwidth}{p{3cm}p{13cm}}%
\par\fontsize{12}{14}\selectfont \textbf{\textcolor{parte}{2. Descripción general}}&El curso de \emph{Sistemas analógicos} colabora en el desarrollo de los siguientes rasgos del plan de estudios: conocer y aplicar los principios de los circuitos eléctricos y la electrónica, y analizar su funcionamiento en las diversas aplicaciónes en ingeniería electromecánica. \newline\newline Los aprendizajes que los estudiantes desarrollarán en el curso son: comprender las características de operación de los amplificadores operacionales con sus topologías más utilizadas; estudiar aplicaciones de IC tales como comparadores, reguladores de voltaje, ADC, convertidores V/F y V/I; comprender las características de selección y operación de los dispositivos electrónicos BJT, MOSFET, IGBT, SCR, TRIACS, MOV usados en la electrónica de potencia; y diseñar aplicaciones electrónicas utilizando circuitos integrados especiales, CI 555, DAC, ADC, entre otros. \newline\newline Para desempeñarse adecuadamente en este curso, los estudiantes deben poner en práctica lo aprendido en el curso de: . \newline\newline Una vez aprobado este curso, los estudiantes podrán emplear algunos de los aprendizajes adquiridos en el curso de: . \\%
\end{tabularx}%
\vspace*{4mm}%
\newline%
\begin{tabularx}{\textwidth}{p{3cm}p{13cm}}%
\par\fontsize{12}{14}\selectfont \textbf{\textcolor{parte}{3. Objetivos}}&Al final del curso la persona estudiante será capaz de:\newline\newline \textbf{Objetivo general}\begin{itemize}\item Analizar y diseñar circuitos electrónicos analógicos de mediana complejidad\end{itemize} \vspace{2mm}\textbf{Objetivos específicos}\begin{itemize}\item Comprender las características de operación de los amplificadores operacionales con sus topologías más utilizadas\item Estudiar aplicaciones de IC tales como comparadores, reguladores de voltaje, ADC, convertidores V/F y V/I\item Comprender las características de selección y operación de los dispositivos electrónicos BJT, MOSFET, IGBT, SCR, TRIACS, MOV usados en la electrónica de potencia\item Diseñar aplicaciones electrónicas utilizando circuitos integrados especiales, CI 555, DAC, ADC, entre otros\end{itemize}\\%
\end{tabularx}%
\newpage%
\begin{tabularx}{\textwidth}{p{3cm}p{13cm}}%
\par\fontsize{12}{14}\selectfont \textbf{\textcolor{parte}{4. Contenidos}}&En el curso se desarrollaran los siguientes temas:\\%
\end{tabularx}%
\newline%
\par \setlength{\leftskip}{4cm} \begin{easylist} \ListProperties(Progressive*=3ex)

& Introducción a los Sistemas Analógicos

&& Conceptos básicos de sistemas analógicos

&& Diferencias entre sistemas analógicos y digitales

& Aplicaciones de los sistemas analógicos

&& Características y funcionamiento de los amplificadores operacionales

&& Analisis de amplificadores Operacionales

&& Topologías más utilizadas: inversor, no inversor, sumador, restador

&& Aplicaciones prácticas de amplificadores operacionales

& Dispositivos Electrónicos

&& Características y operación de BJT, MOSFET, IGBT, SCR, TRIAC

&& Selección de dispositivos para aplicaciones específicas

&& Ejemplos de circuitos con estos dispositivos

& Reguladores de voltaje y comparadores

&& Reguladores de voltaje lineales y conmutados

&& Comparadores y sus aplicaciones

&& Diseño de circuitos con reguladores y comparadores

& Convertidores 

&& Convertidores analógico-digital (ADC) y digital-analógico (DAC).

&& Convertidores V/F y V/I.

&& Integración de sensores en sistemas analógicos.

& Circuitos Integrados Especiales

&& Uso del CI 555 en aplicaciones de temporización y oscilación.

&& Diseño de circuitos con DAC, ADC y otros CI especiales.

&& Ejemplos prácticos y simulaciones.

& Electrónica de Potencia

&& Introducción a la electrónica de potencia.

&& Aplicaciones de dispositivos de potencia en sistemas analógicos.

&& Diseño de circuitos de potencia.

\end{easylist} \setlength{\leftskip}{0cm} %
\newpage%
\par\fontsize{14}{0}\selectfont \textbf{\textcolor{parte}{II parte: Aspectos operativos}}%
\vspace*{4mm}%
\newline%
\fontsize{10}{12}\selectfont %
\begin{tabularx}{\textwidth}{p{3cm}p{13cm}}%
\par\fontsize{12}{14}\selectfont \textbf{\textcolor{parte}{5. Metodología}}&En este curso, la metodología adoptada es constructivista enfocada en la investigación práctica aplicada. Mediante el uso de técnicas como:  estudio de casos, construcción de prototipos y experimentación controlada, se pretende que el estudiante domine la utilización de los componentes y circuitos integrados. Así mismo el facilitador mostrará los contenidos en clases magistrales donde demostrará su aplicación mediante ejemplos, videos y software especializado. \newline\newline \textbf{Las personas estudiantes podrán desarrollar actividades en las que:} \newline\begin{itemize}\item Analizarán y definirán los requisitos del sistema, estableciendo el mejor circuito que solucione el problema planteado.\item Evaluarán distintos circuitos y los compara con el fin de determinar cuál es la mejor alternativa que negocie entre complejidad y error deseado.\item Aplicarán herramientas de simulación para verificar el funcionamiento de la solución planteada\end{itemize}\vspace*{2mm}Este enfoque metodológico permitirá a la persona estudiante analizar y diseñar circuitos electrónicos analógicos de mediana complejidad\vspace*{2mm} \newline  Si un estudiante requiere apoyos educativos, podrá solicitarlos a través del Departamento de Orientación y Psicología. \newline \\%
\end{tabularx}%
\vspace*{2mm}%
\newline%
\begin{tabularx}{\textwidth}{p{3cm}p{13cm}}%
\par\fontsize{12}{14}\selectfont \textbf{\textcolor{parte}{6. Evaluación}}&La evaluación se distribuye en los siguientes rubros: \newline \begin{itemize} \item Tareas: tareas. \item Reportes: reportes. \item Defensa: defensa \end{itemize}\\%
\end{tabularx}%
\vspace*{2mm}%
\newline%
 \begin{minipage}{\linewidth}  \centering  \begin{tabular}{ p{4cm}  p{1.5cm} }  \toprule  Tareas & 20 \% \\  \midrule  Reportes & 60 \% \\  \midrule  Defensa & 20 \% \\  \midrule Total & 100 \% \\  \bottomrule  \end{tabular} \end{minipage}%
\vspace*{2mm}%
\newline%
\begin{tabularx}{\textwidth}{p{3cm}p{13cm}}%
&De conformidad con el artículo 78 del Reglamento del Régimen Enseñanza-Aprendizaje del Instituto Tecnológico de Costa Rica y sus Reformas, en este curso la persona estudiante \textbf{no} tiene derecho a presentar un examen de reposición.\\%
\end{tabularx}%
\vspace*{4mm}%
\newpage%
\begin{tabularx}{\textwidth}{p{3cm}p{13cm}}%
\par\fontsize{12}{14}\selectfont \textbf{\textcolor{parte}{7. Bibliografía}}&\nocite{sedra2020microelectronic} \nocite{boylestad2019electronic} \nocite{horowitz2015art} \nocite{floyd2012fundamentals} \nocite{franco2014design} \\%
\end{tabularx}%
\vspace*{-8mm}\printbibliography[heading=none]%
\begin{tabularx}{\textwidth}{p{3cm}p{13cm}}%
\par\fontsize{12}{14}\selectfont \textbf{\textcolor{parte}{8. Persona docente}}&El curso será impartido por:\\%
\end{tabularx}%
\vspace*{-4mm}\begin{textoMargen}\textbf{Mag. Lisandro Araya Rodriguez} \newline Maestría Ingeniería en Computación. Bachillerato en Ingeniería Electrónica Instituto Tecnológico de Costa Rica. Costa Rica \newline \newline \emph{Correo:} laraya@itcr.ac.cr\emph{  Teléfono:} 0 \vspace*{1mm} \newline \emph{  Oficina:} 19\emph{  Escuela:} Ingeniería Electromecánica\emph{  Sede:} Cartago \vspace*{4mm} \newline \textbf{M.Sc. Nicolás Vaquerano Pineda} \newline Maestría en Electrónica con énfasis en Sistemas Embebidos. Instituto Tecnológico de Costa Rica. Costa Rica \newline \newline \emph{Correo:} nvaquerano@itcr.ac.cr\emph{  Teléfono:} 0 \vspace*{1mm} \newline \emph{  Oficina:} 0\emph{  Escuela:} Ingeniería Electromecánica\emph{  Sede:} Cartago \vspace*{4mm} \newline \end{textoMargen}%
\end{document}