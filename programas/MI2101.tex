\documentclass[letterpaper]{article}%
\usepackage{lastpage}%
\usepackage{parskip}%
\usepackage{geometry}%
\geometry{left=22.5mm,right=16.1mm,top=48mm,bottom=25mm,headheight=12.5mm,footskip=12.5mm}%
\usepackage{fontspec}%
\usepackage[spanish,activeacute]{babel}%
\usepackage{anyfontsize}%
\usepackage{fancyhdr}%
\usepackage{csquotes}%
\usepackage[ampersand]{easylist}%
\usepackage[style=ieee,backend=biber]{biblatex}%
\usepackage[skins,breakable]{tcolorbox}%
\usepackage{color}%
\usepackage{tikz}%
\usepackage{tabularx}%
\usepackage{ragged2e}%
\usepackage{graphicx}%
\usepackage{xcolor}%
%
\setmainfont{Arial}%
\addbibresource{../bibliografia.bib}%
\renewcommand*{\bibfont}{\fontsize{12}{16}\selectfont}%
\definecolor{gris}{rgb}{0.27,0.27,0.27}%
\definecolor{parte}{rgb}{0.02,0.204,0.404}%
\definecolor{azulsuaveTEC}{rgb}{0.02,0.455,0.773}%
\definecolor{fila}{rgb}{0.929,0.929,0.929}%
\definecolor{linea}{rgb}{0.749,0.749,0.749}%
\fancypagestyle{headfoot}{%
\renewcommand{\headrulewidth}{0pt}%
\renewcommand{\footrulewidth}{0pt}%
\fancyhead{%
}%
\fancyfoot{%
}%
\fancyhead[L]{%
\begin{minipage}{0.5\textwidth}%
\flushleft%
\includegraphics[width=62.5mm]{../figuras/Logo.png}%
\end{minipage}%
}%
\fancyfoot[L]{%
\textcolor{azulsuaveTEC}{%
Escuela de Ingeniería Electromecánica%
}%
\par \parbox{0.85\textwidth}{%
\fontsize{8}{0}\selectfont \textcolor{azulsuaveTEC}{Carreras de: Ingeniería en Mantenimiento Industrial, Ingeniería en Electrónica, Ingeniería en Producción Industrial, Ingeniería Mecatrónica, Ingeniería en Materiales}%
}%
}%
\fancyfoot[R]{%
\textcolor{azulsuaveTEC}{%
Página \thepage \hspace{1pt} de \pageref*{LastPage}%
}%
}%
}%
%
\begin{document}%
\normalsize%
\thispagestyle{empty}%
\begin{tikzpicture}[overlay,remember picture]%
\node[inner sep = 0mm,outer sep = 0mm,anchor = north west,xshift = -23mm,yshift = 22mm] at (0.0,0.0) {\includegraphics[width=21cm]{../figuras/Logo_portada.png}};%
\end{tikzpicture}%
\vspace*{100mm}%
\par\fontsize{14}{0}\selectfont \textcolor{black}{Programa del curso MI2101}%
\par\fontsize{18}{25}\selectfont \textbf{\textcolor{azulsuaveTEC}{Dibujo técnico}}%
\vspace*{15mm}%
\newline%
\begin{tabularx}{\textwidth}{m{0.02\textwidth}m{0.98\textwidth}}%
&\hspace*{0mm}\fontsize{12}{0}\selectfont \textbf{\textcolor{gris}{Escuela de Ingeniería Electromecánica}}\\%
[-12pt]%
&\hspace*{0mm}\fontsize{12}{0}\selectfont \textbf{\textcolor{gris}{Carreras de: Ingeniería en Mantenimiento Industrial, Ingeniería en Electrónica, Ingeniería en Producción Industrial, Ingeniería Mecatrónica, Ingeniería en Materiales}}\\%
\end{tabularx}%
\newpage%
\pagestyle{headfoot}%
\par\fontsize{14}{0}\selectfont \textbf{\textcolor{parte}{I parte: Aspectos relativos al plan de estudios}}%
\par\hspace*{2mm}\fontsize{12}{14}\selectfont \textbf{\textcolor{parte}{1. Datos generales}}%
\vspace*{3mm}%
\newline%
\fontsize{10}{12}\selectfont %
\begin{tabularx}{\textwidth}{p{6cm}p{10cm}}%
\textbf{Nombre del curso:}&Dibujo técnico\\%
[10pt]%
\textbf{Código:}&MI2101\\%
[10pt]%
\textbf{Tipo de curso:}&Teórico {-} Práctico\\%
[10pt]%
\textbf{Obligatorio o electivo:}&Obligatorio\\%
[10pt]%
\textbf{Nº de créditos:}&3\\%
[10pt]%
\textbf{Nº horas de clase por semana:}&4\\%
[10pt]%
\textbf{Nº horas extraclase por semana:}&5\\%
[10pt]%
\textbf{Ubicación en el plan de estudios:}&Curso de II semestre en Ingeniería en Mantenimiento IndustrialCurso de I semestre en Ingeniería en Electrónica. Curso de I semestre en Ingeniería en Producción Industrial. Curso de III semestre en Ingeniería Mecatrónica. Curso de IV semestre en Ingeniería en Materiales. \\%
[10pt]%
\textbf{Requisitos:}&Ninguno\\%
[10pt]%
\textbf{Correquisitos:}&Ninguno\\%
[10pt]%
\textbf{El curso es requisito de:}&MI2106 Estática\\%
[10pt]%
\textbf{Asistencia:}&Obligatoria\\%
[10pt]%
\textbf{Suficiencia:}&Si\\%
[10pt]%
\textbf{Posibilidad de reconocimiento:}&Si\\%
[10pt]%
\textbf{Aprobación y actualización del programa:}&I semestre de 2023\\%
[10pt]%
\end{tabularx}%
\newpage%
\begin{tabularx}{\textwidth}{p{3cm}p{13cm}}%
\par\fontsize{12}{14}\selectfont \textbf{\textcolor{parte}{2. Descripción general}}&El curso de Dibujo Técnico contribuye significativamente a la formación y desarrollo profesional de los estudiantes, equipándolos con las habilidades y herramientas necesarias para la comunicación, el diseño y la ejecución en el ámbito de la ingeniería.
\newline%
\newline%
Entre los aprendizajes más destacados se encuentran:  interpretar y aplicar las normas INTE/ISO de dibujo técnico en situaciones prácticas; visualizar y representar objetos tridimensionales en un plano bidimensional y viceversa; desarrollar destrezas en el uso de herramientas CAD para la elaboración de planos; y además de las habilidades técnicas, se busca promover el compromiso, el respeto y la ética profesional entre los participantes.​
\newline%
\newline%
Este curso sienta las bases fundamentales para asignaturas más avanzadas en el campo del diseño y la ingeniería mecánica. Proporciona una comprensión sólida de los principios y técnicas de dibujo técnico que son esenciales para el desarrollo de proyectos más complejos las áreas de diseño y manufactura. De esta manera, establece una conexión con otros cursos de la carrera, preparando a los estudiantes para enfrentar desafíos en su trayectoria académica y profesional.​\\%
\end{tabularx}%
\vspace*{4mm}%
\newline%
\begin{tabularx}{\textwidth}{p{3cm}p{13cm}}%
\par\fontsize{12}{14}\selectfont \textbf{\textcolor{parte}{3. Objetivos}}&Al final del curso la persona estudiante será capaz de:\newline\newline \textbf{Objetivo general}\begin{itemize}\item Realizar un plano de una pieza mecánica, según las normas INTE/ISO, que contenga información necesaria y suficiente para la interpretación de la forma y dimensiones de la pieza.\end{itemize} \vspace{2mm}\textbf{Objetivos específicos}\begin{itemize}\item Aplicar las normas INTE/ISO y de otros estándares.\item Representar mediante proyecciones ortogonales, proyecciones axonométricas, vistas auxiliares y vistas de corte una pieza mecánica.\item Acotar adecuadamente una pieza mecánica en las proyecciones ortogonales.\end{itemize}\\%
\end{tabularx}%
\newpage%
\begin{tabularx}{\textwidth}{p{3cm}p{13cm}}%
\par\fontsize{12}{14}\selectfont \textbf{\textcolor{parte}{4. Contenidos}}&En el curso se desarrollaran los siguientes temas:\\%
\end{tabularx}%


\setlength{\leftskip}{4cm}\begin{easylist}\ListProperties(Progressive*=3ex)

& Generalidades

&& El Dibujo Técnico como lenguaje. Historia del Dibujo Técnico.

&& Objetivos del curso

&& Instrumentos de Dibujo: Escalímetro, escuadras.

&& Rotulado técnico y Formatos para dibujo Técnico

&& Norma de rotulado INTE-ISO 3098/0/2/3-2008

&& Formatos según INTE-ISO 5457-2008

&& Información que debe contener un cajetín INTE-ISO 7200-2008

& Geometría Descriptiva (2 hrs)

&& Consideraciones fundamentales de la Geometría Descriptiva :

&&& Objetivos del curso

&&& Concepto de Geometría Descriptiva y su Historia

&& Proyección del punto, el segmento y los planos en el espacio

&&& Proyección de un punto, el segmento y los planos en las vistas

&&& Proyección de un punto, el segmento y los planos en el espacio

&& Longitudes y dimensiones naturales

&&& Procedimiento de rotación para encontrar la dimensión real de un segmento y un plano;

&&& Procedimiento de sustitución de planos para encontrar la dimensión real de un segmento y un plano;

&&& Procedimiento de superposición para encontrar la dimensión real de un segmento y un plano.

&& Características particulares de la representación de los Cuerpos Geométricos

&&& Características particulares del prisma y su representación;

&&& Características del cono y su representación;

&&& Características de la pirámide y su representación;

&&& Características del cilindro y su representación;

&&& Características del toroide y su representación;

&&& Características de la esfera y su representación

& Escalas (4 hrs)

&& Concepto de escalas

&& Escalas según INTE ISO 5455-2008

& Proyecciones ortogonales (14 hrs)

&& Sistema de proyección y designación de vistas según la norma INTE- ISO 128/30/34-2008

&& Criterios de selección de la vista frontal y la ubicación de las otras vistas.

&& Cantidad de vistas que definen un objeto.

&& Significado y utilización de los tipos de líneas.

&& Tipos de líneas según la norma INTE ISO 128/20/21/22/23/24-2008

& Presentación de un plano:

&& Calidad de líneas

&& Orden y adecuada ubicación de la información

&& Especificaciones técnicas.

&& Práctica de proyecciones ortogonales croquizando “a mano alzada”.

&& Práctica de proyecciones ortogonales utilizando Software de dibujo.

& Proyecciones Axonométricas (12 hrs)

&& Ejes de proyección

&& Tipos de axonometrías

&& Proyecciones axonométricas a mano alzada y utilizando software de dibujo.

& Acotado (4 hrs) 

&& Normas y recomendaciones INTE/ISO 129/1-2008 sobre acotado.

&& Líneas utilizadas en el acotado

&& Posición de la cota

&& Rotulado de cotas

&& Criterios para la acotación correcta de piezas.

&& Relación entre cota y escala.

& Cortes y secciones (4 hrs)

&& Concepto de cortes y secciones. Conveniencia de su utilización

&& Representación e indicación de cortes según la norma INTE/ISO 128/40-44- 2008

&& Achurado

&& Tipos de cortes.

&&& Sección en un plano de corte.

&&& Sección en dos planos paralelos.

&&& Sección en tres planos de corte continuos

&&& Sección en dos planos de intersección

&&& Plano de corte posesionado parcialmente fuera de la pieza

&&& Sección removida de una vista

&&& Secciones sucesivas

&&& Cortes oblicuos o auxiliares

&&& Cortes y secciones parciales

&&& Corte y secciones de piezas simétricas

& Vistas auxiliares simples (4 hrs)

&& Vistas auxiliares

&& Tipos de vistas auxiliares

&& Ubicación de las vistas auxiliares

&& Rotulado de vistas auxiliares.

\end{easylist}\setlength{\leftskip}{0pt}%
\end{document}