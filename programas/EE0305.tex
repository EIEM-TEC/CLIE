\documentclass[letterpaper]{article}%
\usepackage{lastpage}%
\usepackage{parskip}%
\usepackage{geometry}%
\geometry{left=22.5mm,right=16.1mm,top=48mm,bottom=25mm,headheight=12.5mm,footskip=12.5mm}%
\usepackage{fontspec}%
\usepackage[spanish,activeacute]{babel}%
\usepackage{anyfontsize}%
\usepackage{fancyhdr}%
\usepackage{csquotes}%
\usepackage[ampersand]{easylist}%
\usepackage[style=ieee,backend=biber]{biblatex}%
\usepackage[skins,breakable]{tcolorbox}%
\usepackage{booktabs}%
\usepackage{color}%
\usepackage{tikz}%
\usepackage{tabularx}%
\usepackage{ragged2e}%
\usepackage{graphicx}%
\usepackage{xcolor}%
%
\setmainfont{Arial}%
\addbibresource{../bibliografia.bib}%
\renewcommand*{\bibfont}{\fontsize{10}{14}\selectfont}%

\defbibenvironment{bibliography}
    {\list
    {\printfield[labelnumberwidth]{labelnumber}}
    {\setlength{\leftmargin}{4cm}
    \setlength{\rightmargin}{1.1cm}
    \setlength{\itemindent}{0pt}
    \setlength{\itemsep}{\bibitemsep}
    \setlength{\parsep}{\bibparsep}}}
    {\endlist}
{\item}
%

\newenvironment{textoMargen}
    {%
    \begin{list}{}{%
        \setlength{\leftmargin}{3.6cm}%
        \setlength{\rightmargin}{1.1cm}%
    }%
    \item[]%
  }
  {%
    \end{list}%
  }
%
\definecolor{gris}{rgb}{0.27,0.27,0.27}%
\definecolor{parte}{rgb}{0.02,0.204,0.404}%
\definecolor{azulsuaveTEC}{rgb}{0.02,0.455,0.773}%
\definecolor{fila}{rgb}{0.929,0.929,0.929}%
\definecolor{linea}{rgb}{0.749,0.749,0.749}%
\fancypagestyle{headfoot}{%
\renewcommand{\headrulewidth}{0pt}%
\renewcommand{\footrulewidth}{0pt}%
\fancyhead{%
}%
\fancyfoot{%
}%
\fancyhead[L]{%
\begin{minipage}{0.5\textwidth}%
\flushleft%
\includegraphics[width=62.5mm]{../figuras/Logo.png}%
\end{minipage}%
}%
\fancyfoot[R]{%
\textcolor{black}{%
Página \thepage \hspace{1pt} de \pageref*{LastPage}%
}%
}%
}%
%
\begin{document}%
\normalsize%
\thispagestyle{empty}%
\begin{tikzpicture}[overlay,remember picture]%
\node[inner sep = 0mm,outer sep = 0mm,anchor = north west,xshift = -23mm,yshift = 22mm] at (0.0,0.0) {\includegraphics[width=21cm]{../figuras/Logo_portada.png}};%
\end{tikzpicture}%
\vspace*{100mm}%
\par\fontsize{14}{0}\selectfont \textcolor{black}{Programa del curso EE{-}0305}%
\par\fontsize{18}{25}\selectfont \textbf{\textcolor{black}{Transductores}}%
\vspace*{15mm}%
\newline%
\begin{tabularx}{\textwidth}{m{0.02\textwidth}m{0.98\textwidth}}%
&\hspace*{0mm}\fontsize{12}{0}\selectfont \textbf{\textcolor{gris}{Escuela de Ingeniería Electromecánica}}\\%
[-12pt]%
&\hspace*{0mm}\fontsize{12}{0}\selectfont \textbf{\textcolor{gris}{Carrera de Ingeniería Electromecánica}}\\%
\end{tabularx}%
\newpage%
\pagestyle{headfoot}%
\par\fontsize{14}{0}\selectfont \textbf{\textcolor{parte}{I parte: Aspectos relativos al plan de estudios}}%
\par\hspace*{2mm}\fontsize{12}{14}\selectfont \textbf{\textcolor{parte}{1. Datos generales}}%
\vspace*{3mm}%
\newline%
\fontsize{10}{12}\selectfont %
\begin{tabularx}{\textwidth}{p{6cm}p{10cm}}%
\textbf{Nombre del curso:}&Transductores\\%
[10pt]%
\textbf{Código:}&EE{-}0305\\%
[10pt]%
\textbf{Tipo de curso:}&Teórico {-} Práctico\\%
[10pt]%
\textbf{Obligatorio o electivo:}&Obligatorio\\%
[10pt]%
\textbf{Nº de créditos:}&2\\%
[10pt]%
\textbf{Nº horas de clase por semana:}&3\\%
[10pt]%
\textbf{Nº horas extraclase por semana:}&3\\%
[10pt]%
\textbf{Ubicación en el plan de estudios:}&Curso de 3\textsuperscript{er} semestre en Ingeniería Electromecánica\\%
[10pt]%
\textbf{Requisitos:}&CA{-}2026 Introducción a la computación\\%
[10pt]%
\textbf{Correquisitos:}&EE{-}0304 Laboratorio de circuitos I\\%
[10pt]%
\textbf{El curso es requisito de:}&EE{-}0405 Instrumentación\\%
[10pt]%
\textbf{Asistencia:}&Obligatoria\\%
[10pt]%
\textbf{Suficiencia:}&No\\%
[10pt]%
\textbf{Posibilidad de reconocimiento:}&Si\\%
[10pt]%
\textbf{Aprobación y actualización del programa:}&01/01/2026 en sesión de Consejo de Escuela 01{-}2026\\%
[10pt]%
\end{tabularx}%
\newpage%
\begin{tabularx}{\textwidth}{p{3cm}p{13cm}}%
\par\fontsize{12}{14}\selectfont \textbf{\textcolor{parte}{2. Descripción general}}&El curso de \emph{Transductores} colabora en el desarrollo de los siguientes rasgos del plan de estudios: implementar sistemas de instrumentación para la medición y modificación de variables físicas en sistemas electromecánicos; y aplicar principios de metrología para medir variables físicas en sistemas electromecánicos. \newline\newline Los aprendizajes que los estudiantes desarrollarán en el curso son: comprender las características estáticas, dinámicas, eléctricas y de fabricación de los transductores; seleccionar transductores según su aplicación en sistemas específicos, considerando sus características y principios de funcionamiento; y experimentar con transductores mediante prácticas que permitan aprendizajes significativos y el desarrollo de habilidades aplicadas. \newline\newline Para desempeñarse adecuadamente en este curso, los estudiantes deben poner en práctica lo aprendido en los cursos de: Física general I, y Física general II. \newline\newline Una vez aprobado este curso, los estudiantes podrán emplear algunos de los aprendizajes adquiridos en los cursos de: Instrumentación, Modelado y simulación de sistemas, Control automático, y Control por eventos discretos. \\%
\end{tabularx}%
\vspace*{4mm}%
\newline%
\begin{tabularx}{\textwidth}{p{3cm}p{13cm}}%
\par\fontsize{12}{14}\selectfont \textbf{\textcolor{parte}{3. Objetivos}}&Al final del curso la persona estudiante será capaz de:\newline\newline \textbf{Objetivo general}\begin{itemize}\item Evaluar transductores para su integración en sistemas de instrumentación dedicados a la medición y modificación de variables físicas en sistemas electromecánicos\end{itemize} \vspace{2mm}\textbf{Objetivos específicos}\begin{itemize}\item Comprender las características estáticas, dinámicas, eléctricas y de fabricación de los transductores\item Seleccionar transductores según su aplicación en sistemas específicos, considerando sus características y principios de funcionamiento\item Experimentar con transductores mediante prácticas que permitan aprendizajes significativos y el desarrollo de habilidades aplicadas\end{itemize}\\%
\end{tabularx}%
\newpage%
\begin{tabularx}{\textwidth}{p{3cm}p{13cm}}%
\par\fontsize{12}{14}\selectfont \textbf{\textcolor{parte}{4. Contenidos}}&En el curso se desarrollaran los siguientes temas:\\%
\end{tabularx}%
\newline%
\par \setlength{\leftskip}{4cm} \begin{easylist} \ListProperties(Progressive*=3ex)

& Conceptos básicos

&& Señales, estímulos y sistemas

&& Modelos y simulaciones

&& Sensores, actuadores y transductores

&& Clasificaciones

& Características de los transductores 

&& Función de transferencia

&& Entrada y salida a escala completa

&& Exactitud y precisión

&& Repetibilidad y reproducibilidad

&& Histéresis y no linealidad

&& Saturación y banda muerta

&& Resolución

&& Impedancia de salida

&& Excitación

&& Características dinámicas 

&& Confiabilidad e incertidumbre

& Transductores térmicos

&& Bimetales

&& Termoresistivos

&& Termoeléctricos

&& Termomecánicos

&& Inductivos y microondas para calentamiento

& Transductores ópticos

&& Fotoconductores

&& Fotodiodos 

&& Fototransistores

&& Fotovoltaicos

&& Piroeléctricos y termopilas para radiación térmica

& Transductores eléctricos y magnéticos

&& Capacitivos

&& Magnetrostrictivos

&& Piezoelectricos

&& Piezoresistivos

&& Efecto Hall

&& Motores

&& Solenoides

& Transductores acústicos

&& Micrófonos e hidrófonos

&& Parlantes

&& Ultrasónicos

& Transductores químicos

&& Electroquímicos

&& Potenciométricos

&& Termoquímicos

& Transductores de radiación

&& Ionizante

&& Microondas

& Transductores MEMS

&& Métodos de fabricación

&& Unidades de medición inercial (IMU)

&& Sensores de presión

&& Micrófonos 

&& Interruptores ópticos

& Interfaces de los transductores

&& Amplificadores operacionales

&& Amplificadores de potencia

&& PWMs para actuadores

&& Convertidores A/D y D/A

&& Puentes

&& Transmisión de datos

&& Excitadores

&& Ruido e interferencia

\end{easylist} \setlength{\leftskip}{0cm} %
\newpage%
\par\fontsize{14}{0}\selectfont \textbf{\textcolor{parte}{II parte: Aspectos operativos}}%
\vspace*{4mm}%
\newline%
\fontsize{10}{12}\selectfont %
\begin{tabularx}{\textwidth}{p{3cm}p{13cm}}%
\par\fontsize{12}{14}\selectfont \textbf{\textcolor{parte}{5. Metodología}}&En este curso, se utilizará el enfoque sistémico-complejo para la ejecución de las sesiones magistrales y se integrará la investigación práctica aplicada para las sesiones prácticas. Esta última se implementará mediante técnicas como la experimentación controlada y el estudio de casos.\newline\newline \textbf{Las personas estudiantes podrán desarrollar actividades en las que:} \newline\begin{itemize}\item Recibirán instrucción sobre los principios físicos que gobiernan el comportamiento de los transductores.\item Analizarán alternativas para seleccionar el transductor adecuado de acuerdo con cada aplicación vista en los estudios de caso.\end{itemize}\vspace*{2mm}Este enfoque metodológico permitirá a la persona estudiante evaluar transductores para su integración en sistemas de instrumentación dedicados a la medición y modificación de variables físicas en sistemas electromecánicos\vspace*{2mm} \newline  Si un estudiante requiere apoyos educativos, podrá solicitarlos a través del Departamento de Orientación y Psicología. \newline \\%
\end{tabularx}%
\vspace*{2mm}%
\newline%
\begin{tabularx}{\textwidth}{p{3cm}p{13cm}}%
\par\fontsize{12}{14}\selectfont \textbf{\textcolor{parte}{6. Evaluación}}&La evaluación se distribuye en los siguientes rubros: \newline \begin{itemize} \item Tareas: investigación sobre temas relacionados con principios físicos y aplicaciones. \item Pruebas cortas: evaluación del correcto análisis y selección de transductores basados en estudios de casos. \item Reportes: desarrollo y conclusión de los experimentos relacionados con los temas de características de los transductores. \end{itemize}\\%
\end{tabularx}%
\vspace*{2mm}%
\newline%
 \begin{minipage}{\linewidth}  \centering  \begin{tabular}{ p{4cm}  p{1.5cm} }  \toprule  Tareas & 20 \% \\  \midrule  Pruebas cortas & 20 \% \\  \midrule  Reportes & 60 \% \\  \midrule Total & 100 \% \\  \bottomrule  \end{tabular} \end{minipage}%
\vspace*{2mm}%
\newline%
\begin{tabularx}{\textwidth}{p{3cm}p{13cm}}%
&De conformidad con el artículo 78 del Reglamento del Régimen Enseñanza-Aprendizaje del Instituto Tecnológico de Costa Rica y sus Reformas, en este curso la persona estudiante \textbf{no} tiene derecho a presentar un examen de reposición.\\%
\end{tabularx}%
\vspace*{4mm}%
\newline%
\begin{tabularx}{\textwidth}{p{3cm}p{13cm}}%
\par\fontsize{12}{14}\selectfont \textbf{\textcolor{parte}{7. Bibliografía}}&\nocite{ida2020sensors} \nocite{fraden2016sensors} \nocite{pallas2012sensors} \\%
\end{tabularx}%
\vspace*{-8mm}\printbibliography[heading=none]%
\begin{tabularx}{\textwidth}{p{3cm}p{13cm}}%
\par\fontsize{12}{14}\selectfont \textbf{\textcolor{parte}{8. Persona docente}}&El curso será impartido por:\\%
\end{tabularx}%
\vspace*{-4mm}\begin{textoMargen}\textbf{Dr.{-}Ing. Luis Diego Murillo Soto} \newline Máster en Ciencias de la Ingeniería Eléctrica. Universidad de Costa Rica. Costa Rica \newline \newline  Máster en computación. Ingeniero en Mantenimiento Industrial. Tecnológico de Costa Rica.Costa Rica \newline \newline  Técnico en Electrónica. COVAO \newline \newline \emph{Correo:} lmurillo@itcr.ac.cr\emph{  Teléfono:} 25509347 \vspace*{1mm} \newline \emph{  Oficina:} 7\emph{  Escuela:} Ingeniería Electromecánica\emph{  Sede:} Cartago \vspace*{4mm} \newline \textbf{Dr.{-}Ing. Juan José Rojas Hernández} \newline Doctor en ciencia aplicada a la integración de sistemas. Instituto Tecnológico de Kyushu. Japón. \newline \newline  Máster en electrónica con énfasis en microsistemas. Licenciado en Mantenimiento Industrial. Instituto Tecnológico de Costa Rica. Costa Rica. \newline \newline \emph{Correo:} juan.rojas@itcr.ac.cr\emph{  Teléfono:} 88581419 \vspace*{1mm} \newline \emph{  Oficina:} 31\emph{  Escuela:} Ingeniería Electromecánica\emph{  Sede:} Cartago \vspace*{4mm} \newline \end{textoMargen}%
\end{document}