\documentclass[letterpaper]{article}%
\usepackage{lastpage}%
\usepackage{parskip}%
\usepackage{geometry}%
\geometry{left=22.5mm,right=16.1mm,top=48mm,bottom=25mm,headheight=12.5mm,footskip=12.5mm}%
\usepackage{fontspec}%
\usepackage[spanish,activeacute]{babel}%
\usepackage{anyfontsize}%
\usepackage{fancyhdr}%
\usepackage{csquotes}%
\usepackage[ampersand]{easylist}%
\usepackage[style=ieee,backend=biber]{biblatex}%
\usepackage[skins,breakable]{tcolorbox}%
\usepackage{color}%
\usepackage{tikz}%
\usepackage{tabularx}%
\usepackage{ragged2e}%
\usepackage{graphicx}%
\usepackage{xcolor}%
%
\setmainfont{Arial}%
\addbibresource{../bibliografia.bib}%
\renewcommand*{\bibfont}{\fontsize{12}{16}\selectfont}%
\definecolor{gris}{rgb}{0.27,0.27,0.27}%
\definecolor{parte}{rgb}{0.02,0.204,0.404}%
\definecolor{azulsuaveTEC}{rgb}{0.02,0.455,0.773}%
\definecolor{fila}{rgb}{0.929,0.929,0.929}%
\definecolor{linea}{rgb}{0.749,0.749,0.749}%
\fancypagestyle{headfoot}{%
\renewcommand{\headrulewidth}{0pt}%
\renewcommand{\footrulewidth}{0pt}%
\fancyhead{%
}%
\fancyfoot{%
}%
\fancyhead[L]{%
\begin{minipage}{0.5\textwidth}%
\flushleft%
\includegraphics[width=62.5mm]{../figuras/Logo.png}%
\end{minipage}%
}%
\fancyfoot[L]{%
\textcolor{azulsuaveTEC}{%
Escuela de Ingeniería Electromecánica%
}%
\par \parbox{0.85\textwidth}{%
\fontsize{8}{0}\selectfont \textcolor{azulsuaveTEC}{Carrera de Ingeniería Electromecánica con énfasis en sistemas ciberfísicos}%
}%
}%
\fancyfoot[R]{%
\textcolor{azulsuaveTEC}{%
Página \thepage \hspace{1pt} de \pageref*{LastPage}%
}%
}%
}%
%
\begin{document}%
\normalsize%
\thispagestyle{empty}%
\begin{tikzpicture}[overlay,remember picture]%
\node[inner sep = 0mm,outer sep = 0mm,anchor = north west,xshift = -23mm,yshift = 22mm] at (0.0,0.0) {\includegraphics[width=21cm]{../figuras/Logo_portada.png}};%
\end{tikzpicture}%
\vspace*{100mm}%
\par\fontsize{14}{0}\selectfont \textcolor{black}{Programa del curso MI1001}%
\par\fontsize{18}{25}\selectfont \textbf{\textcolor{azulsuaveTEC}{Ingeniería de sistemas}}%
\vspace*{15mm}%
\newline%
\begin{tabularx}{\textwidth}{m{0.02\textwidth}m{0.98\textwidth}}%
&\hspace*{0mm}\fontsize{12}{0}\selectfont \textbf{\textcolor{gris}{Escuela de Ingeniería Electromecánica}}\\%
[-12pt]%
&\hspace*{0mm}\fontsize{12}{0}\selectfont \textbf{\textcolor{gris}{Carrera de Ingeniería Electromecánica con énfasis en sistemas ciberfísicos}}\\%
\end{tabularx}%
\newpage%
\pagestyle{headfoot}%
\par\fontsize{14}{0}\selectfont \textbf{\textcolor{parte}{I parte: Aspectos relativos al plan de estudios}}%
\par\hspace*{2mm}\fontsize{12}{14}\selectfont \textbf{\textcolor{parte}{1. Datos generales}}%
\vspace*{3mm}%
\newline%
\fontsize{10}{12}\selectfont %
\begin{tabularx}{\textwidth}{p{6cm}p{10cm}}%
\textbf{Nombre del curso:}&Ingeniería de sistemas\\%
[10pt]%
\textbf{Código:}&MI1001\\%
[10pt]%
\textbf{Tipo de curso:}&Teórico\\%
[10pt]%
\textbf{Obligatorio o electivo:}&Obligatorio\\%
[10pt]%
\textbf{Nº de créditos:}&3\\%
[10pt]%
\textbf{Nº horas de clase por semana:}&4\\%
[10pt]%
\textbf{Nº horas extraclase por semana:}&5\\%
[10pt]%
\textbf{Ubicación en el plan de estudios:}&Curso de VIII semestre en Ingeniería Electromecánica con énfasis en sistemas ciberfísicos\\%
[10pt]%
\textbf{Requisitos:}&MI0714 Administración de proyectos\\%
[10pt]%
\textbf{Correquisitos:}&Ninguno\\%
[10pt]%
\textbf{El curso es requisito de:}&MI1004 Modelado numérico y simulación computacional\\%
[10pt]%
\textbf{Asistencia:}&Obligatoria\\%
[10pt]%
\textbf{Suficiencia:}&Si\\%
[10pt]%
\textbf{Posibilidad de reconocimiento:}&Si\\%
[10pt]%
\textbf{Aprobación y actualización del programa:}&I semestre de 2026\\%
[10pt]%
\end{tabularx}%
\newpage%
\begin{tabularx}{\textwidth}{p{3cm}p{13cm}}%
\par\fontsize{12}{14}\selectfont \textbf{\textcolor{parte}{2. Descripción general}}&El curso de Ingeniería de sistemas contribuye a que los estudiantes puedan analizar, diseñar y gestionar sistemas complejos que integren componentes físicos y digitales; además de liderar equipos de trabajo promoviendo el pensamiento crítico, la colaboración y la innovación, fomentando una convivencia respetuosa e inclusiva
\newline%
\newline%
Los aprendizajes que los estudiantes desarrollarán en el curso son: definir los requisitos y la arquitectura de los sistemas considerando las necesidades y expectativas de los interesados; integrar componentes físicos y digitales en un sistema coherente y funcional; verificar el diseño del sistema o sus partes por medio de modelos, simulaciones y/o prototipos; y colaborar en equipos de trabajo multidisciplinarios en el análisis, diseño y gestión de sistemas.
\newline%
\newline%
Para desempeñarse adecuadamente en este curso, los estudiantes deben poner en práctica lo aprendido en los cursos de: Metodología de la investigación y Administración de proyectos.
\newline%
\newline%
Una vez aprobado este curso, los estudiantes podrán emplear algunos de los aprendizajes adquiridos en los cursos de: Modelado numérico y simulación computacional y Taller de integración de sistemas.\\%
\end{tabularx}%
\vspace*{4mm}%
\newline%
\begin{tabularx}{\textwidth}{p{3cm}p{13cm}}%
\par\fontsize{12}{14}\selectfont \textbf{\textcolor{parte}{3. Objetivos}}&Al final del curso la persona estudiante será capaz de:\newline\newline \textbf{Objetivo general}\begin{itemize}\item Integrar principios, metodologías y herramientas de la ingeniería de sistemas en el análisis, diseño y gestión de sistemas, colaborando con equipos de trabajo multidisciplinarios.\end{itemize} \vspace{2mm}\textbf{Objetivos específicos}\begin{itemize}\item Definir los requisitos y la arquitectura de los sistemas considerando las necesidades y expectativas de los interesados.\item Integrar componentes físicos y digitales en un sistema coherente y funcional.\item Verificar el diseño del sistema o sus partes por medio de modelos, simulaciones y/o prototipos.\item Colaborar en equipos de trabajo multidisciplinarios en el análisis, diseño y gestión de sistemas.\end{itemize}\\%
\end{tabularx}%
\newpage%
\begin{tabularx}{\textwidth}{p{3cm}p{13cm}}%
\par\fontsize{12}{14}\selectfont \textbf{\textcolor{parte}{4. Contenidos}}&En el curso se desarrollaran los siguientes temas:\\%
\end{tabularx}%


\setlength{\leftskip}{4cm}\begin{easylist}\ListProperties(Progressive*=3ex)

& Introducción a la Ingeniería de Sistemas

&& Conceptos fundamentales de la ingeniería de sistemas

&& Ciclo de vida del desarrollo de sistemas

&& Enfoque sistémico y pensamiento complejo en la resolución de problemas

& Trabajo colaborativo en equipos multidisciplinarios

&& Definición de roles

&& Estrategias de trabajo en equipos multidisciplinarios

&& Gestión de la información y la toma de decisiones

&& Planificación del trabajo y gestión de tareas

&& Control de versiones

& Definición y análisis de requisitos

&& Identificación de interesados y sus expectativas

&& Técnicas para la recopilación y análisis de requisitos

&& Diferentes niveles de requisitos y su relación con la arquitectura del sistema

&& Modelado de requisitos y especificaciones funcionales

& Diseño de la arquitectura 

&& Principios y enfoques para el diseño de la arquitectura del sistema

&& Metodologías de evaluación de configuraciones del sistema (trade-off)

&& Gestión de interfaces y compatibilidad de sistemas

&& Estrategias para la integración de subsistemas en un sistema coherente

& Modelado, simulación y verificación del Diseño

&& Métodos de modelado para la representación de sistemas

&& Desarrollo de representaciones funcionales del sistema

&& Evaluación de desempeño y optimización de sistemas

&& Análisis de sensibilidad y pruebas iterativas en entornos virtuales

&& Verificación del sistema en escenarios simulados

&& Validación de requisitos

\end{easylist}\setlength{\leftskip}{0pt}%
\newpage%
\par\fontsize{14}{0}\selectfont \textbf{\textcolor{parte}{II parte: Aspectos operativos}}%
\vspace*{4mm}%
\newline%
\fontsize{10}{12}\selectfont %
\begin{tabularx}{\textwidth}{p{3cm}p{13cm}}%
\par\fontsize{12}{14}\selectfont \textbf{\textcolor{parte}{5. Metodología}}&En este curso, la estrategia central será la investigación práctica aplicada, implementada mediante técnicas como el aprendizaje basado en proyectos (PBL), el análisis de alternativas (trade-off) y el modelado y simulación.\newline\newline \textbf{El curso contempla:}\begin{itemize}\item Definir los requisitos y la arquitectura de los sistemas considerando las necesidades y expectativas de los interesados.\item Integrar componentes físicos y digitales en un sistema coherente y funcional.\item Verificar el diseño del sistema o sus partes por medio de modelos, simulaciones y/o prototipos.\item Colaborar en equipos de trabajo multidisciplinarios en el análisis, diseño y gestión de sistemas.\end{itemize}\\%
\end{tabularx}%
\newline%
\end{document}