\documentclass[letterpaper]{article}%
\usepackage{lastpage}%
\usepackage{parskip}%
\usepackage{geometry}%
\geometry{left=22.5mm,right=16.1mm,top=48mm,bottom=25mm,headheight=12.5mm,footskip=12.5mm}%
\usepackage{fontspec}%
\usepackage[spanish,activeacute]{babel}%
\usepackage{anyfontsize}%
\usepackage{fancyhdr}%
\usepackage{csquotes}%
\usepackage[ampersand]{easylist}%
\usepackage[style=ieee,backend=biber]{biblatex}%
\usepackage[skins,breakable]{tcolorbox}%
\usepackage{booktabs}%
\usepackage{color}%
\usepackage{tikz}%
\usepackage{tabularx}%
\usepackage{ragged2e}%
\usepackage{graphicx}%
\usepackage{xcolor}%
%
\setmainfont{Arial}%
\addbibresource{../bibliografia.bib}%
\renewcommand*{\bibfont}{\fontsize{10}{14}\selectfont}%

\defbibenvironment{bibliography}
    {\list
    {\printfield[labelnumberwidth]{labelnumber}}
    {\setlength{\leftmargin}{4cm}
    \setlength{\rightmargin}{1.1cm}
    \setlength{\itemindent}{0pt}
    \setlength{\itemsep}{\bibitemsep}
    \setlength{\parsep}{\bibparsep}}}
    {\endlist}
{\item}
%

\newenvironment{textoMargen}
    {%
    \begin{list}{}{%
        \setlength{\leftmargin}{3.6cm}%
        \setlength{\rightmargin}{1.1cm}%
    }%
    \item[]%
  }
  {%
    \end{list}%
  }
%
\definecolor{gris}{rgb}{0.27,0.27,0.27}%
\definecolor{parte}{rgb}{0.02,0.204,0.404}%
\definecolor{azulsuaveTEC}{rgb}{0.02,0.455,0.773}%
\definecolor{fila}{rgb}{0.929,0.929,0.929}%
\definecolor{linea}{rgb}{0.749,0.749,0.749}%
\fancypagestyle{headfoot}{%
\renewcommand{\headrulewidth}{0pt}%
\renewcommand{\footrulewidth}{0pt}%
\fancyhead{%
}%
\fancyfoot{%
}%
\fancyhead[L]{%
\begin{minipage}{0.5\textwidth}%
\flushleft%
\includegraphics[width=62.5mm]{../figuras/Logo.png}%
\end{minipage}%
}%
\fancyfoot[R]{%
\textcolor{black}{%
Página \thepage \hspace{1pt} de \pageref*{LastPage}%
}%
}%
}%
%
\begin{document}%
\normalsize%
\thispagestyle{empty}%
\begin{tikzpicture}[overlay,remember picture]%
\node[inner sep = 0mm,outer sep = 0mm,anchor = north west,xshift = -23mm,yshift = 22mm] at (0.0,0.0) {\includegraphics[width=21cm]{../figuras/Logo_portada.png}};%
\end{tikzpicture}%
\vspace*{100mm}%
\par\fontsize{14}{0}\selectfont \textcolor{black}{Programa del curso PI{-}0502}%
\par\fontsize{18}{25}\selectfont \textbf{\textcolor{black}{Estadística aplicada}}%
\vspace*{15mm}%
\newline%
\begin{tabularx}{\textwidth}{m{0.02\textwidth}m{0.98\textwidth}}%
&\hspace*{0mm}\fontsize{12}{0}\selectfont \textbf{\textcolor{gris}{Escuela de Ingeniería Electromecánica}}\\%
[-12pt]%
&\hspace*{0mm}\fontsize{12}{0}\selectfont \textbf{\textcolor{gris}{Carrera de Ingeniería Electromecánica}}\\%
\end{tabularx}%
\newpage%
\pagestyle{headfoot}%
\par\fontsize{14}{0}\selectfont \textbf{\textcolor{parte}{I parte: Aspectos relativos al plan de estudios}}%
\par\hspace*{2mm}\fontsize{12}{14}\selectfont \textbf{\textcolor{parte}{1. Datos generales}}%
\vspace*{3mm}%
\newline%
\fontsize{10}{12}\selectfont %
\begin{tabularx}{\textwidth}{p{6cm}p{10cm}}%
\textbf{Nombre del curso:}&Estadística aplicada\\%
[10pt]%
\textbf{Código:}&PI{-}0502\\%
[10pt]%
\textbf{Tipo de curso:}&Teórico\\%
[10pt]%
\textbf{Obligatorio o electivo:}&Obligatorio\\%
[10pt]%
\textbf{Nº de créditos:}&2\\%
[10pt]%
\textbf{Nº horas de clase por semana:}&3\\%
[10pt]%
\textbf{Nº horas extraclase por semana:}&3\\%
[10pt]%
\textbf{Ubicación en el plan de estudios:}&Curso de 5\textsuperscript{to} semestre en Ingeniería Electromecánica\\%
[10pt]%
\textbf{Requisitos:}&MA{-}2104 Cálculo superior\\%
[10pt]%
\textbf{Correquisitos:}&Ninguno\\%
[10pt]%
\textbf{El curso es requisito de:}&EE{-}0602 Fiabilidad y disponibilidad de sistemas electromecánicos\\%
[10pt]%
\textbf{Asistencia:}&Libre\\%
[10pt]%
\textbf{Suficiencia:}&Si\\%
[10pt]%
\textbf{Posibilidad de reconocimiento:}&Si\\%
[10pt]%
\textbf{Aprobación y actualización del programa:}&01/01/2026 en sesión de Consejo de Escuela 01{-}2026\\%
[10pt]%
\end{tabularx}%
\newpage%
\begin{tabularx}{\textwidth}{p{3cm}p{13cm}}%
\par\fontsize{12}{14}\selectfont \textbf{\textcolor{parte}{2. Descripción general}}&El curso de \emph{Estadística aplicada} colabora en el desarrollo de los siguientes rasgos del plan de estudios: aplicar herramientas estadísticas para diseñar experimentos, evaluar datos con rigor científico, y garantizar la confiabilidad, disponibilidad, mantenibilidad y seguridad en sistemas electromecánicos. \newline\newline Los aprendizajes que los estudiantes desarrollarán en el curso son: diseñar experimentos considerando principios estadísticos para garantizar la validez de los datos obtenidos; analizar datos experimentales mediante técnicas estadísticas para extraer información relevante; aplicar fundamentos de probabilidad esenciales para la interpretación de experimentos y pruebas de hipótesis; y utilizar software especializado para el procesamiento y análisis de datos en estudios experimentales. \newline\newline Para desempeñarse adecuadamente en este curso, los estudiantes deben poner en práctica lo aprendido en los cursos de: Matemática general, y Fiabilidad y disponibilidad de sistemas electromecánicos. \newline\newline Una vez aprobado este curso, los estudiantes podrán emplear algunos de los aprendizajes adquiridos en el curso de: Administración de proyectos. \\%
\end{tabularx}%
\vspace*{4mm}%
\newline%
\begin{tabularx}{\textwidth}{p{3cm}p{13cm}}%
\par\fontsize{12}{14}\selectfont \textbf{\textcolor{parte}{3. Objetivos}}&Al final del curso la persona estudiante será capaz de:\newline\newline \textbf{Objetivo general}\begin{itemize}\item Aplicar metodologías de diseño de experimentos y herramientas estadísticas para la obtención y el análisis de datos en ingeniería\end{itemize} \vspace{2mm}\textbf{Objetivos específicos}\begin{itemize}\item Diseñar experimentos considerando principios estadísticos para garantizar la validez de los datos obtenidos\item Analizar datos experimentales mediante técnicas estadísticas para extraer información relevante\item Aplicar fundamentos de probabilidad esenciales para la interpretación de experimentos y pruebas de hipótesis\item Utilizar software especializado para el procesamiento y análisis de datos en estudios experimentales\end{itemize}\\%
\end{tabularx}%
\newpage%
\begin{tabularx}{\textwidth}{p{3cm}p{13cm}}%
\par\fontsize{12}{14}\selectfont \textbf{\textcolor{parte}{4. Contenidos}}&En el curso se desarrollaran los siguientes temas:\\%
\end{tabularx}%
\newline%
\par \setlength{\leftskip}{4cm} \begin{easylist} \ListProperties(Progressive*=3ex)

& Fundamentos de estadística aplicada

&& Introducción a la estadística en ingeniería

&& Tipos de datos y escalas de medición

&& Organización y presentación de datos

&& Medidas de tendencia central

&& Medidas de dispersión 

& Fundamentos de probabilidad para Diseño de Experimentos

&& Concepto de variabilidad y aleatoriedad

&& Distribución normal

&& Distribución t-Student

&& Distribución chi-cuadrado

&& Distribución Weibull

&& Uso de tablas estadísticas y software para cálculos

& Análisis exploratorio de datos

&& Histogramas, boxplots y gráficos de dispersión

&& Identificación de valores atípicos y tendencias

&& Correlación y regresión lineal simple

&& Análisis de relaciones entre variables

& Pruebas de hipótesis y comparación de medias

&& Concepto de prueba de hipótesis y errores tipo I y II

&& Pruebas de hipótesis para la media de una población

&& Comparación de dos medias

&& Pruebas de hipótesis para la varianza

& Diseño de Experimentos y Análisis de Varianza (ANOVA)

&& Introducción al diseño de experimentos

&& Diseño completamente aleatorizado

&& Diseño en bloques

&& Análisis de varianza (ANOVA) de un solo factor

&& Aplicaciones del ANOVA en ingeniería

& Regresión y modelado estadístico

&& Regresión lineal simple y múltiple

&& Supuestos del modelo de regresión

&& Evaluación de la bondad del ajuste

&& Aplicaciones en ingeniería

& Aplicación de Software para el Análisis de Datos

&& Introducción a herramientas estadísticas

&& Generación de gráficos y tablas descriptivas

&& Implementación de pruebas de hipótesis

&& Modelado de datos y regresión

\end{easylist} \setlength{\leftskip}{0cm} %
\newpage%
\par\fontsize{14}{0}\selectfont \textbf{\textcolor{parte}{II parte: Aspectos operativos}}%
\vspace*{4mm}%
\newline%
\fontsize{10}{12}\selectfont %
\begin{tabularx}{\textwidth}{p{3cm}p{13cm}}%
\par\fontsize{12}{14}\selectfont \textbf{\textcolor{parte}{5. Metodología}}&En este curso, se utilizará el enfoque sistémico-complejo para la ejecución de las sesiones magistrales y se integrará la investigación práctica aplicada para las asignaciones. Esta última se implementará mediante técnicas como el estudio de casos.\newline\newline \textbf{Las personas estudiantes podrán desarrollar actividades en las que:} \newline\begin{itemize}\item Diseñarán experimentos a la medida basados en estudios de caso.\item Analizarán datos experimentales aplicando principios estadísticos.\item Utilizarán software especializado para procesar y analizar datos.\end{itemize}\vspace*{2mm}Este enfoque metodológico permitirá a la persona estudiante aplicar metodologías de diseño de experimentos y herramientas estadísticas para la obtención y el análisis de datos en ingeniería\vspace*{2mm} \newline  Si un estudiante requiere apoyos educativos, podrá solicitarlos a través del Departamento de Orientación y Psicología. \newline \\%
\end{tabularx}%
\vspace*{2mm}%
\newline%
\begin{tabularx}{\textwidth}{p{3cm}p{13cm}}%
\par\fontsize{12}{14}\selectfont \textbf{\textcolor{parte}{6. Evaluación}}&La evaluación se distribuye en los siguientes rubros: \newline \begin{itemize} \item Examenes parciales: evaluaciones formales que miden el nivel de comprensión y aplicación de los conceptos clave del curso. Generalmente cubren una parte significativa del contenido visto hasta la fecha y pueden incluir problemas teóricos y prácticos. \item Pruebas cortas: evaluaciones breves y frecuentes que sirven para comprobar el dominio de temas específicos. Suelen ser de menor peso en la calificación final y permiten reforzar el aprendizaje continuo. \item Tareas: actividad asignada a los estudiantes con el propósito de reforzar, aplicar o evaluar el aprendizaje de un tema específico. Puede requerir investigación, resolución de problemas, desarrollo de habilidades prácticas o aplicación de conocimientos teóricos. \end{itemize}\\%
\end{tabularx}%
\vspace*{2mm}%
\newline%
 \begin{minipage}{\linewidth}  \centering  \begin{tabular}{ p{4cm}  p{1.5cm} }  \toprule  Examenes parciales & 60 \% \\  \midrule  Pruebas cortas & 20 \% \\  \midrule  Tareas & 20 \% \\  \midrule Total & 100 \% \\  \bottomrule  \end{tabular} \end{minipage}%
\vspace*{2mm}%
\newline%
\begin{tabularx}{\textwidth}{p{3cm}p{13cm}}%
&De conformidad con el artículo 78 del Reglamento del Régimen Enseñanza-Aprendizaje del Instituto Tecnológico de Costa Rica y sus Reformas, en este curso la persona estudiante  tiene derecho a presentar un examen de reposición si su nota luego de redondeo es 60 o 65.\\%
\end{tabularx}%
\vspace*{4mm}%
\newline%
\begin{tabularx}{\textwidth}{p{3cm}p{13cm}}%
\par\fontsize{12}{14}\selectfont \textbf{\textcolor{parte}{7. Bibliografía}}&\nocite{montgomery2020applied} \\%
\end{tabularx}%
\vspace*{-8mm}\printbibliography[heading=none]%
\begin{tabularx}{\textwidth}{p{3cm}p{13cm}}%
\par\fontsize{12}{14}\selectfont \textbf{\textcolor{parte}{8. Persona docente}}&El curso será impartido por:\\%
\end{tabularx}%
\vspace*{-4mm}\begin{textoMargen}\textbf{Mag. Rellenar} \newline Rellenar \newline \newline \emph{Correo:} xxx@itcr.ac.cr\emph{  Teléfono:} 0 \vspace*{1mm} \newline \emph{  Oficina:} 0\emph{  Escuela:} Ingeniería en Producción Industrial\emph{  Sede:} Cartago \vspace*{4mm} \newline \end{textoMargen}%
\end{document}